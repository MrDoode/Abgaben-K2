\section{Durchführung} 
Als erstes wurde die Enzymlösung mit \qty{204,7}{\milli\gram} Urease und \qty{10}{\milli\liter} destilliertem Wasser erstellt. 
Anschließend wurde eine  Verdünnungsreihe mit einer Harnstofflösung hergestellt. 
Verwendet wurden jeweils \qty{1}{\milli\liter}, \qty{2}{\milli\liter}, \qty{3}{\milli\liter}, \qty{5}{\milli\liter}, \qty{25}{\milli\liter} und \qty{40}{\milli\liter} einer \qty{0,1}{\mole\per\liter} Harnstofflösung. 
Es wurde beim zweiten Messkolben (\qty{2}{\milli\liter}) etwa \qty{2}{\milli\meter} zu viel destilliertes Wasser hinzugegeben. 
Als Nächstes wurde die Elektrode fixiert und der Magnetrührer eingeschaltet. 
Es wurde drauf geachtet, die Rührgeschwindigkeit konstant zu halten. 
Anschließend wurden die 6 Messungen durchgeführt, wobei jede mit \qty{1}{\milli\liter} Enzymlösung gestartet wurde. 
Anschließend wurde die unkatalysierte Reaktion untersucht. 
Dabei wurde zuerst das Wasserbad auf \qty{50}{\degreeCelsius} erhitzt. Danach wurden ca. \qty{50}{\milli\liter} einer \qty{0,5}{\mole\per\liter} Harnstofflösung auf \qty{48}{\degreeCelsius} temperiert und die  Messung gestartet. 
Nach \qty{10}{min} wurde die Messung beendet. 
Es wurde versäumt, am Ende der Messung \qty{1}{\milli\liter} Urease dazuzugeben, um die drastische Zunahme der Reaktionsgeschwindigkeit zu beobachten. 
Zum Schluss wurde zur Kalibrierung \qty{4}{\milli\mole\per\liter} Ammoniumkarbonat für \qty{50}{\second} gemessen.
