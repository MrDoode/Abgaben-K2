\section{Theoretischer Hintergrund}
Bei der Hydrolyse von Harnstoff wirkt das Enzym Urease wie ein Katalysator.
Bei der Reaktion entsteht Ammoniumkarbonat, welches im Wasser dissoziiert vorliegt.

$$ H_3N-CO-NH_3 + 2(H_2O) \hspace{3mm}\underrightarrow{Urease}\hspace{3mm} {(H_4N)_2CO_3}_{(aq)}$$

Die Leitfähigkeit der Lösung ist damit proportional zu dem Reaktionsfortschritt.
Über die Bestimmung der Leitfähigkeit einer Lösung mit bekannter Konzentration an Ammoniumkarbonat kann so die Konzentration einer Lösung mit anderer Leitfähigkeit nach Gleichung \ref{UmrechnungLeitfähigkeitKonzentration} bestimmt werden.
\begin{equation}
	C_l = \kappa_l\cdot \frac{c_{Norm}}{\kappa_{Norm}}
\label{UmrechnungLeitfähigkeitKonzentration}
\end{equation}
Über mehrere Messungen können nun die Reaktionsgeschwindigkeiten bei verschiedenen Konzentrationen bestimmt werden.
Es ergibt sich ein Michaelis-Menten-Zusammenhang zwischen Anfangsgeschwindigkeit und Konzentration (Gleichung \ref{MichaelisMenten}).
\begin{equation}
	v=v_0\cdot\frac{[s]}{[s]+K_M}
\label{MichaelisMenten}\\
\end{equation}
\begin{equation}
	\frac{1}{V_0} = \frac{1}{V'_0} + \frac{K_M}{V'_0}\cdot \frac{1}{C{H,0}}
\label{LineweaverBurk}\\
\end{equation}
Nach Lineweaver-Burk (Gleichung \ref{LineweaverBurk}) gibt die reziproke Auftragung der Anfangsgeschwindigkeiten eine Gerade, deren Abzissenabschnitt, $-K_M$ entspricht, während der Ordinatenabschnitt $\frac{1}{V_0}$ entspricht.
Für $c_{H,0} << K_M$ kann die Geschwindigkeitskonstante dieses Versuchs nun über Gleichung \ref{K_V1} bestimmt werden.
\begin{equation}
	v_0=K_2\cdot c_{U,0}
	\label{K_V1}
\end{equation}
Anschließend wird die Konzentrationsänderung des Ammoniumkarbonats über die Zeit beobachtet. 
Nach dem Geschwindigkeitsgesetz 1. Ordnung (Gleichung \ref{gesch.gesetz}) ergibt die Auftragung von $\ln\left(\frac{[A_0]}{[A_t]}\right)$ über die Zeit eine Gerade deren Steigung der Geschwindigkeitskonstante entspricht.
\begin{equation}
	\ln{\left(\frac{[A_0]}{[A_t]}\right)=kt}
\label{gesch.gesetz}
\end{equation}
