\documentclass[a4paper, 11pt]{article}

\usepackage{fullpage}
%\usepackage[inline]{showlabels}
\usepackage[german]{babel}
\usepackage[use-xspace,per-mode=fraction]{siunitx}
\usepackage{gnuplot-lua-tikz}
\usepackage{caption}
\usepackage{float}
\usepackage{amsmath}
\usepackage{chemfig}
\usepackage{biblatex}
\usepackage{easyReview}


\renewcommand{\arraystretch}{1.5}

\bibliography{Protokoll.bib}

\title{Praktikum der Physikalischen Chemie\\\large Hydrolyse von Harnstoff}
\author{\\Samed Hür (huer@uni-bremen.de)\\Janosch Ehlers (jaeh@uni-bremen.de)\\\small Gruppe H\\ \\ \textbf{Betreuer:} Tobias Borrmann (tobias.borrmann@uni-bremen.de)}
\date{21.11.2022}

\begin{document}
\maketitle
\thispagestyle{empty}
\newpage

\tableofcontents
\thispagestyle{empty}
\newpage
\clearpage
\pagenumbering{arabic} 

\section{Einleitung}
Ziel ist die Bestimmung der Geschwindigkeitskonstante bei der ureasekatalysierten und unkatalysierten Umsetzung von Harnstoff in Ammoniumkarbonat durch Messung der elektrischen Leitfähigkeit über die Zeit.\alert{Voll scheiße das ding}

\section{Theoretischer Hintergrund}
Bei der Hydrolyse von Harnstoff wirkt das Enzym Urease wie ein Katalysator.
Bei der Reaktion entsteht Ammoniumkarbonat, welches im Wasser dissoziiert vorliegt.

$$ H_3N-CO-NH_3 + 2(H_2O) \hspace{3mm}\underrightarrow{Urease}\hspace{3mm} {(H_4N)_2CO_3}_{(aq)}$$

Die Leitfähigkeit der Lösung ist damit proportional zu dem Reaktionsfortschritt.
Über die Bestimmung der Leitfähigkeit einer Lösung mit bekannter Konzentration an Ammoniumkarbonat kann so die Konzentration einer Lösung mit anderer Leitfähigkeit nach Gleichung \ref{UmrechnungLeitfähigkeitKonzentration} bestimmt werden.
\begin{equation}
	C_l = \kappa_l\cdot \frac{c_{Norm}}{\kappa_{Norm}}
\label{UmrechnungLeitfähigkeitKonzentration}
\end{equation}
Über mehrere Messungen können nun die Reaktionsgeschwindigkeiten bei verschiedenen Konzentrationen bestimmt werden.
Es ergibt sich ein Michaelis-Menten-Zusammenhang zwischen Anfangsgeschwindigkeit und Konzentration (Gleichung \ref{MichaelisMenten}).
\begin{equation}
	v=v_0\cdot\frac{[s]}{[s]+K_M}
\label{MichaelisMenten}\\
\end{equation}
\begin{equation}
	\frac{1}{V_0} = \frac{1}{V'_0} + \frac{K_M}{V'_0}\cdot \frac{1}{C{H,0}}
\label{LineweaverBurk}\\
\end{equation}
Nach Lineweaver-Burk (Gleichung \ref{LineweaverBurk}) gibt die reziproke Auftragung der Anfangsgeschwindigkeiten eine Gerade, deren Abzissenabschnitt, $-K_M$ entspricht, während der Ordinatenabschnitt $\frac{1}{V_0}$ entspricht.
Für $c_{H,0} << K_M$ kann die Geschwindigkeitskonstante dieses Versuchs nun über Gleichung \ref{K_V1} bestimmt werden.
\begin{equation}
	v_0=K_2\cdot c_{U,0}
	\label{K_V1}
\end{equation}
Anschließend wird die Konzentrationsänderung des Ammoniumkarbonats über die Zeit beobachtet. 
Nach dem Geschwindigkeitsgesetz 1. Ordnung (Gleichung \ref{gesch.gesetz}) ergibt die Auftragung von $\ln\left(\frac{[A_0]}{[A_t]}\right)$ über die Zeit eine Gerade deren Steigung der Geschwindigkeitskonstante entspricht.
\begin{equation}
	\ln{\left(\frac{[A_0]}{[A_t]}\right)=kt}
\label{gesch.gesetz}
\end{equation}

\input{Durchführung.tex}
\section{Auswertung}
Bei der Auswertung sind wir auf erhebliche Unstimmigkeiten unserer Daten gestoßen.
Im weiteren Verlauf sind konnte keine Michaelis-Menten Beziehung hergestellt werden.
Die Auswertungsstrategie kann nicht angewandt werden, somit sind unsere Daten Unbrauchbar.\\ 
Ausgewertet werden also die Daten unserer Partnergruppe.\\
\subsection{Anfangsgeschwindigkeitsbestimmung}
Die ersten 10 Messpunkte nach Hinzugabe des Enzyms sind in Tabelle \ref{Daten_V1} zu sehen.
Es wurden in jedem Experiment mehr Daten aufgezeichnet als hier angegeben, allerdings im weiteren Verlauf nicht ausgewertet.
Wie in Abbildung \ref{BSP_V0_Bestimmung} zu sehen, wurde anschließend für jede Messung eine Ausgleichsgerade für die ersten sechs Messpunkt und die Messpunkte zwei bis sieben berechnet.
Für die Umrechnung von Leitfähigkeit zu Konzentration wurde Gleichung \ref{UmrechnungLeitfähigkeitKonzentration} verwendet.
Die Leitfähigkeit $\kappa_{Norm}$ hat hier den Wert des Durchschnitts aller erhaltenen Leitfähigkeiten der Kalibrierlösung.
Beispielhaft ergibt sich also für den ersten Messwert des ersten Messkolbens: 
$$ \qty{4,5}{\micro\siemens\per\meter} \cdot \frac{\qty{4}{\milli\mole\per\liter}}{\qty{545,4}{\micro\siemens\per\meter}} = \qty{0,033}{\milli\mole\per\liter} $$
Durch die Darstellung von Konzentration über die Zeit ist die erhaltene Steigung dieser Geraden direkt die Anfangsgeschwindigkeit dieser Messung. 
Die beiden Werte spiegeln hier also eine angenommene obere und untere Grenze der Anfangsgeschwindigkeit wieder.
Da Leit\-fähig\-keits\-mess\-gerät und Kalibrierlösung als fehlerfrei angesehen werden, ist auch die Konzentration fehlerfrei.
\begin{figure}[t]
\centering
\resizebox{0.6\linewidth}{!}{%
\begin{tikzpicture}[gnuplot]
%% generated with GNUPLOT 5.4p5 (Lua 5.4; terminal rev. Jun 2020, script rev. 115)
%% Sa 19 Nov 2022 14:25:56 CET
\path (0.000,0.000) rectangle (12.500,8.750);
\gpcolor{color=gp lt color border}
\gpsetlinetype{gp lt border}
\gpsetdashtype{gp dt solid}
\gpsetlinewidth{1.00}
\draw[gp path] (1.748,0.616)--(1.928,0.616);
\draw[gp path] (11.947,0.616)--(11.767,0.616);
\node[gp node right] at (1.564,0.616) {0.00002};
\draw[gp path] (1.748,1.399)--(1.928,1.399);
\draw[gp path] (11.947,1.399)--(11.767,1.399);
\node[gp node right] at (1.564,1.399) {0.00003};
\draw[gp path] (1.748,2.181)--(1.928,2.181);
\draw[gp path] (11.947,2.181)--(11.767,2.181);
\node[gp node right] at (1.564,2.181) {0.00003};
\draw[gp path] (1.748,2.964)--(1.928,2.964);
\draw[gp path] (11.947,2.964)--(11.767,2.964);
\node[gp node right] at (1.564,2.964) {0.00004};
\draw[gp path] (1.748,3.746)--(1.928,3.746);
\draw[gp path] (11.947,3.746)--(11.767,3.746);
\node[gp node right] at (1.564,3.746) {0.00004};
\draw[gp path] (1.748,4.529)--(1.928,4.529);
\draw[gp path] (11.947,4.529)--(11.767,4.529);
\node[gp node right] at (1.564,4.529) {0.00005};
\draw[gp path] (1.748,5.311)--(1.928,5.311);
\draw[gp path] (11.947,5.311)--(11.767,5.311);
\node[gp node right] at (1.564,5.311) {0.00005};
\draw[gp path] (1.748,6.093)--(1.928,6.093);
\draw[gp path] (11.947,6.093)--(11.767,6.093);
\node[gp node right] at (1.564,6.093) {0.00006};
\draw[gp path] (1.748,6.876)--(1.928,6.876);
\draw[gp path] (11.947,6.876)--(11.767,6.876);
\node[gp node right] at (1.564,6.876) {0.00006};
\draw[gp path] (1.748,7.659)--(1.928,7.659);
\draw[gp path] (11.947,7.659)--(11.767,7.659);
\node[gp node right] at (1.564,7.659) {0.00007};
\draw[gp path] (1.748,8.441)--(1.928,8.441);
\draw[gp path] (11.947,8.441)--(11.767,8.441);
\node[gp node right] at (1.564,8.441) {0.00007};
\draw[gp path] (1.748,0.616)--(1.748,0.796);
\draw[gp path] (1.748,8.441)--(1.748,8.261);
\node[gp node center] at (1.748,0.308) {$-5$};
\draw[gp path] (2.881,0.616)--(2.881,0.796);
\draw[gp path] (2.881,8.441)--(2.881,8.261);
\node[gp node center] at (2.881,0.308) {$0$};
\draw[gp path] (4.014,0.616)--(4.014,0.796);
\draw[gp path] (4.014,8.441)--(4.014,8.261);
\node[gp node center] at (4.014,0.308) {$5$};
\draw[gp path] (5.148,0.616)--(5.148,0.796);
\draw[gp path] (5.148,8.441)--(5.148,8.261);
\node[gp node center] at (5.148,0.308) {$10$};
\draw[gp path] (6.281,0.616)--(6.281,0.796);
\draw[gp path] (6.281,8.441)--(6.281,8.261);
\node[gp node center] at (6.281,0.308) {$15$};
\draw[gp path] (7.414,0.616)--(7.414,0.796);
\draw[gp path] (7.414,8.441)--(7.414,8.261);
\node[gp node center] at (7.414,0.308) {$20$};
\draw[gp path] (8.547,0.616)--(8.547,0.796);
\draw[gp path] (8.547,8.441)--(8.547,8.261);
\node[gp node center] at (8.547,0.308) {$25$};
\draw[gp path] (9.681,0.616)--(9.681,0.796);
\draw[gp path] (9.681,8.441)--(9.681,8.261);
\node[gp node center] at (9.681,0.308) {$30$};
\draw[gp path] (10.814,0.616)--(10.814,0.796);
\draw[gp path] (10.814,8.441)--(10.814,8.261);
\node[gp node center] at (10.814,0.308) {$35$};
\draw[gp path] (11.947,0.616)--(11.947,0.796);
\draw[gp path] (11.947,8.441)--(11.947,8.261);
\node[gp node center,rotate=-270] at (-0.15,4.713) {$c / \frac{mol}{l}$};
\node[gp node center] at (6.9,-0.2) {$t / s$};
\node[gp node center] at (11.947,0.308) {$40$};
\draw[gp path] (1.748,8.441)--(1.748,0.616)--(11.947,0.616)--(11.947,8.441)--cycle;
\node[gp node right] at (10.479,8.107) {Data};
\gpcolor{rgb color={0.580,0.000,0.827}}
\gpsetpointsize{4.00}
\gp3point{gp mark 1}{}{(4.014,2.651)}
\gp3point{gp mark 1}{}{(5.148,3.455)}
\gp3point{gp mark 1}{}{(6.281,4.028)}
\gp3point{gp mark 1}{}{(7.414,4.716)}
\gp3point{gp mark 1}{}{(8.547,5.291)}
\gp3point{gp mark 1}{}{(9.681,5.865)}
\gp3point{gp mark 1}{}{(10.814,6.439)}
\gp3point{gp mark 1}{}{(11.121,8.107)}
\gpcolor{color=gp lt color border}
\node[gp node right] at (10.479,7.799) {v(c)=$8,130e-5\cdot c+2,953e-5$};
\gpcolor{rgb color={0.000,0.620,0.451}}
\draw[gp path] (10.663,7.799)--(11.579,7.799);
\draw[gp path] (1.748,1.471)--(1.851,1.529)--(1.954,1.587)--(2.057,1.645)--(2.160,1.703)%
  --(2.263,1.760)--(2.366,1.818)--(2.469,1.876)--(2.572,1.934)--(2.675,1.992)--(2.778,2.050)%
  --(2.881,2.107)--(2.984,2.165)--(3.087,2.223)--(3.190,2.281)--(3.293,2.339)--(3.396,2.397)%
  --(3.499,2.454)--(3.602,2.512)--(3.705,2.570)--(3.808,2.628)--(3.911,2.686)--(4.014,2.744)%
  --(4.117,2.801)--(4.220,2.859)--(4.324,2.917)--(4.427,2.975)--(4.530,3.033)--(4.633,3.091)%
  --(4.736,3.148)--(4.839,3.206)--(4.942,3.264)--(5.045,3.322)--(5.148,3.380)--(5.251,3.438)%
  --(5.354,3.495)--(5.457,3.553)--(5.560,3.611)--(5.663,3.669)--(5.766,3.727)--(5.869,3.785)%
  --(5.972,3.842)--(6.075,3.900)--(6.178,3.958)--(6.281,4.016)--(6.384,4.074)--(6.487,4.132)%
  --(6.590,4.189)--(6.693,4.247)--(6.796,4.305)--(6.899,4.363)--(7.002,4.421)--(7.105,4.479)%
  --(7.208,4.536)--(7.311,4.594)--(7.414,4.652)--(7.517,4.710)--(7.620,4.768)--(7.723,4.826)%
  --(7.826,4.883)--(7.929,4.941)--(8.032,4.999)--(8.135,5.057)--(8.238,5.115)--(8.341,5.173)%
  --(8.444,5.230)--(8.547,5.288)--(8.650,5.346)--(8.753,5.404)--(8.856,5.462)--(8.959,5.520)%
  --(9.062,5.577)--(9.165,5.635)--(9.268,5.693)--(9.371,5.751)--(9.475,5.809)--(9.578,5.867)%
  --(9.681,5.924)--(9.784,5.982)--(9.887,6.040)--(9.990,6.098)--(10.093,6.156)--(10.196,6.214)%
  --(10.299,6.271)--(10.402,6.329)--(10.505,6.387)--(10.608,6.445)--(10.711,6.503)--(10.814,6.561)%
  --(10.917,6.618)--(11.020,6.676)--(11.123,6.734)--(11.226,6.792)--(11.329,6.850)--(11.432,6.908)%
  --(11.535,6.965)--(11.638,7.023)--(11.741,7.081)--(11.844,7.139)--(11.947,7.197);
\gpcolor{color=gp lt color border}
\node[gp node right] at (10.479,7.491) {$v(c)=7,67e-7\cdot c+3,05e-5$};
\gpcolor{rgb color={0.337,0.706,0.914}}
\draw[gp path] (10.663,7.491)--(11.579,7.491);
\draw[gp path] (1.748,1.659)--(1.851,1.714)--(1.954,1.768)--(2.057,1.823)--(2.160,1.877)%
  --(2.263,1.932)--(2.366,1.986)--(2.469,2.041)--(2.572,2.096)--(2.675,2.150)--(2.778,2.205)%
  --(2.881,2.259)--(2.984,2.314)--(3.087,2.368)--(3.190,2.423)--(3.293,2.477)--(3.396,2.532)%
  --(3.499,2.587)--(3.602,2.641)--(3.705,2.696)--(3.808,2.750)--(3.911,2.805)--(4.014,2.859)%
  --(4.117,2.914)--(4.220,2.969)--(4.324,3.023)--(4.427,3.078)--(4.530,3.132)--(4.633,3.187)%
  --(4.736,3.241)--(4.839,3.296)--(4.942,3.350)--(5.045,3.405)--(5.148,3.460)--(5.251,3.514)%
  --(5.354,3.569)--(5.457,3.623)--(5.560,3.678)--(5.663,3.732)--(5.766,3.787)--(5.869,3.842)%
  --(5.972,3.896)--(6.075,3.951)--(6.178,4.005)--(6.281,4.060)--(6.384,4.114)--(6.487,4.169)%
  --(6.590,4.223)--(6.693,4.278)--(6.796,4.333)--(6.899,4.387)--(7.002,4.442)--(7.105,4.496)%
  --(7.208,4.551)--(7.311,4.605)--(7.414,4.660)--(7.517,4.715)--(7.620,4.769)--(7.723,4.824)%
  --(7.826,4.878)--(7.929,4.933)--(8.032,4.987)--(8.135,5.042)--(8.238,5.096)--(8.341,5.151)%
  --(8.444,5.206)--(8.547,5.260)--(8.650,5.315)--(8.753,5.369)--(8.856,5.424)--(8.959,5.478)%
  --(9.062,5.533)--(9.165,5.588)--(9.268,5.642)--(9.371,5.697)--(9.475,5.751)--(9.578,5.806)%
  --(9.681,5.860)--(9.784,5.915)--(9.887,5.969)--(9.990,6.024)--(10.093,6.079)--(10.196,6.133)%
  --(10.299,6.188)--(10.402,6.242)--(10.505,6.297)--(10.608,6.351)--(10.711,6.406)--(10.814,6.460)%
  --(10.917,6.515)--(11.020,6.570)--(11.123,6.624)--(11.226,6.679)--(11.329,6.733)--(11.432,6.788)%
  --(11.535,6.842)--(11.638,6.897)--(11.741,6.952)--(11.844,7.006)--(11.947,7.061);
\gpcolor{color=gp lt color border}
\draw[gp path] (1.748,8.441)--(1.748,0.616)--(11.947,0.616)--(11.947,8.441)--cycle;
%% coordinates of the plot area
\gpdefrectangularnode{gp plot 1}{\pgfpoint{1.748cm}{0.616cm}}{\pgfpoint{11.947cm}{8.441cm}}
\end{tikzpicture}
%% gnuplot variables
}
\caption{Zunahme von Ammoniumkarbonat über die Zeit für Maßkolben 1 (\qty{1}{\milli\liter} Harnstoff)}
\label{BSP_V0_Bestimmung}
\end{figure}

\begin{table}[b]
  \centering
\vspace{5mm}
  \begin{tabular}{l|cccccc}
t/\unit{\second}& $\kappa_{1}/\unit{\micro\siemens\per\meter}$	& $\kappa_{2}/\unit{\micro\siemens\per\meter}$ & $\kappa_{3}/\unit{\micro\siemens\per\meter}$ & $\kappa_{5}/\unit{\micro\siemens\per\meter}$ & $\kappa_{25}/\unit{\micro\siemens\per\meter}$ & $\kappa_{40}/\unit{\micro\siemens\per\meter}$\\
\vspace{-3mm}& \\
\hline
\hline
5 & 4,5 & 2,5 & 4,6 & 2,8 & 2,7 & 7,8\\
10 & 5,2 & 5,3 & 6,7 & 6,5 & 8,2 & 11,7\\
15 & 5,7 & 7,5 & 8,6 & 9,4 & 12,2 & 15,4\\
20 & 6,3 & 9,5 & 10,3 & 12,1 & 16 & 19,1\\
25 & 6,8 & 11,2 & 11,8 & 14,6 & 19,8 & 22,7\\
30 & 7,3 & 12,9 & 13,4 & 17,1 & 23,5 & 26,3\\
35 & 7,8 & 14,5 & 14,8 & 19,6 & 27,2 & 29,8\\
40 & 8,2 & 16,1 & 16,2 & 21,9 & 30,8 & 33,3\\
45 & 8,7 & 17,6 & 17,6 & 24,2 & 34,3 & 36,7\\
50 & 9,1 & 19 & 19 & 26,5 & 37,8 & 40\\
    \hline& 
  \end{tabular}
  \caption{Leitfähigkeiten der Verdünnungsreihe nach Enzym-Hinzugabe}
  \label{Daten_V1}
\end{table}





\subsection{Herstellung der Michaelis-Menten Beziehung}
Die errechneten Anfangsgeschwindigkeiten, werden nun in Abhängigkeit der Eduktkonzentration Dargestellt.
Zusehen ist dies in Abbildung \ref{MM-Beziehung}.
\begin{figure}[t]
\centering
\begin{tikzpicture}[gnuplot]
%% generated with GNUPLOT 5.4p5 (Lua 5.4; terminal rev. Jun 2020, script rev. 115)
%% Mo 21 Nov 2022 22:49:57 CET
\path (0.000,0.000) rectangle (12.500,8.750);
\gpcolor{color=gp lt color border}
\gpsetlinetype{gp lt border}
\gpsetdashtype{gp dt solid}
\gpsetlinewidth{1.00}
\draw[gp path] (1.872,0.985)--(2.052,0.985);
\draw[gp path] (11.947,0.985)--(11.767,0.985);
\node[gp node right] at (1.688,0.985) {$0$};
\draw[gp path] (1.872,2.050)--(2.052,2.050);
\draw[gp path] (11.947,2.050)--(11.767,2.050);
\node[gp node right] at (1.688,2.050) {$1\times10^{-6}$};
\draw[gp path] (1.872,3.115)--(2.052,3.115);
\draw[gp path] (11.947,3.115)--(11.767,3.115);
\node[gp node right] at (1.688,3.115) {$2\times10^{-6}$};
\draw[gp path] (1.872,4.180)--(2.052,4.180);
\draw[gp path] (11.947,4.180)--(11.767,4.180);
\node[gp node right] at (1.688,4.180) {$3\times10^{-6}$};
\draw[gp path] (1.872,5.246)--(2.052,5.246);
\draw[gp path] (11.947,5.246)--(11.767,5.246);
\node[gp node right] at (1.688,5.246) {$4\times10^{-6}$};
\draw[gp path] (1.872,6.311)--(2.052,6.311);
\draw[gp path] (11.947,6.311)--(11.767,6.311);
\node[gp node right] at (1.688,6.311) {$5\times10^{-6}$};
\draw[gp path] (1.872,7.376)--(2.052,7.376);
\draw[gp path] (11.947,7.376)--(11.767,7.376);
\node[gp node right] at (1.688,7.376) {$6\times10^{-6}$};
\draw[gp path] (1.872,8.441)--(2.052,8.441);
\draw[gp path] (11.947,8.441)--(11.767,8.441);
\node[gp node right] at (1.688,8.441) {$7\times10^{-6}$};
\draw[gp path] (1.872,0.985)--(1.872,1.165);
\draw[gp path] (1.872,8.441)--(1.872,8.261);
\node[gp node center] at (1.872,0.677) {$0$};
\draw[gp path] (3.131,0.985)--(3.131,1.165);
\draw[gp path] (3.131,8.441)--(3.131,8.261);
\node[gp node center] at (3.131,0.677) {$0.01$};
\draw[gp path] (4.391,0.985)--(4.391,1.165);
\draw[gp path] (4.391,8.441)--(4.391,8.261);
\node[gp node center] at (4.391,0.677) {$0.02$};
\draw[gp path] (5.650,0.985)--(5.650,1.165);
\draw[gp path] (5.650,8.441)--(5.650,8.261);
\node[gp node center] at (5.650,0.677) {$0.03$};
\draw[gp path] (6.910,0.985)--(6.910,1.165);
\draw[gp path] (6.910,8.441)--(6.910,8.261);
\node[gp node center] at (6.910,0.677) {$0.04$};
\draw[gp path] (8.169,0.985)--(8.169,1.165);
\draw[gp path] (8.169,8.441)--(8.169,8.261);
\node[gp node center] at (8.169,0.677) {$0.05$};
\draw[gp path] (9.428,0.985)--(9.428,1.165);
\draw[gp path] (9.428,8.441)--(9.428,8.261);
\node[gp node center] at (9.428,0.677) {$0.06$};
\draw[gp path] (10.688,0.985)--(10.688,1.165);
\draw[gp path] (10.688,8.441)--(10.688,8.261);
\node[gp node center] at (10.688,0.677) {$0.07$};
\draw[gp path] (11.947,0.985)--(11.947,1.165);
\draw[gp path] (11.947,8.441)--(11.947,8.261);
\node[gp node center] at (11.947,0.677) {$0.08$};
\draw[gp path] (1.872,8.441)--(1.872,0.985)--(11.947,0.985)--(11.947,8.441)--cycle;
\node[gp node center,rotate=-270] at (-0.2,4.713) {$v / \frac{mol}{L\cdot s}$};
\node[gp node center] at (6.909,0.215) {$c / \frac{mol}{L}$};
\node[gp node right] at (5.100,8.107) {Obergrenze};
\gpcolor{rgb color={0.580,0.000,0.827}}
\gpsetpointsize{4.00}
\gp3point{gp mark 1}{}{(2.119,1.851)}
\gp3point{gp mark 1}{}{(2.613,3.708)}
\gp3point{gp mark 1}{}{(3.107,5.382)}
\gp3point{gp mark 1}{}{(8.045,7.351)}
\gp3point{gp mark 1}{}{(11.749,6.922)}
\gp3point{gp mark 1}{}{(5.642,8.107)}
\gpcolor{color=gp lt color border}
\node[gp node right] at (5.100,7.799) {Untergrenze};
\gpcolor{rgb color={0.000,0.620,0.451}}
\gp3point{gp mark 2}{}{(2.119,1.802)}
\gp3point{gp mark 2}{}{(2.613,3.503)}
\gp3point{gp mark 2}{}{(3.107,5.052)}
\gp3point{gp mark 2}{}{(8.045,6.909)}
\gp3point{gp mark 2}{}{(11.749,6.766)}
\gp3point{gp mark 2}{}{(5.642,7.799)}
\gpcolor{color=gp lt color border}
\node[gp node right] at (5.100,7.491) {Ungenutzte Daten};
\gpcolor{rgb color={0.337,0.706,0.914}}
\gp3point{gp mark 3}{}{(2.366,4.186)}
\gp3point{gp mark 3}{}{(2.366,3.837)}
\gp3point{gp mark 3}{}{(5.642,7.491)}
\gpcolor{color=gp lt color border}
\draw[gp path] (1.872,8.441)--(1.872,0.985)--(11.947,0.985)--(11.947,8.441)--cycle;
%% coordinates of the plot area
\gpdefrectangularnode{gp plot 1}{\pgfpoint{1.872cm}{0.985cm}}{\pgfpoint{11.947cm}{8.441cm}}
\end{tikzpicture}
%% gnuplot variables

\caption{Michaelis-Menten Beziehung}
\label{MM-Beziehung}
\end{figure}
Damit nun die Maximalgeschwindigkeit und die Michaelis-Menten Konstante bestimmt werden kann, muss zudem auch die Doppeltreziproke Visualisierung dieser Beziehung erstellt werden (Abb. \ref{LB_Auftragung}).
Die Ausgleichsgeraden der Daten in dieser Darstellung gibt nach Gleichung \ref{LineweaverBurk}, den Zusammenhang mit den gesuchten Größen.

\begin{figure}[t]
\centering
\begin{tikzpicture}[gnuplot]
%% generated with GNUPLOT 5.4p5 (Lua 5.4; terminal rev. Jun 2020, script rev. 115)
%% Sa 19 Nov 2022 15:40:48 CET
\path (0.000,0.000) rectangle (12.500,8.750);
\gpcolor{color=gp lt color border}
\gpsetlinetype{gp lt border}
\gpsetdashtype{gp dt solid}
\gpsetlinewidth{1.00}
\draw[gp path] (2.056,0.985)--(2.236,0.985);
\draw[gp path] (11.947,0.985)--(11.767,0.985);
\node[gp node right] at (1.872,0.985) {$0$};
\draw[gp path] (2.056,2.050)--(2.236,2.050);
\draw[gp path] (11.947,2.050)--(11.767,2.050);
\node[gp node right] at (1.872,2.050) {$200000$};
\draw[gp path] (2.056,3.115)--(2.236,3.115);
\draw[gp path] (11.947,3.115)--(11.767,3.115);
\node[gp node right] at (1.872,3.115) {$400000$};
\draw[gp path] (2.056,4.180)--(2.236,4.180);
\draw[gp path] (11.947,4.180)--(11.767,4.180);
\node[gp node right] at (1.872,4.180) {$600000$};
\draw[gp path] (2.056,5.246)--(2.236,5.246);
\draw[gp path] (11.947,5.246)--(11.767,5.246);
\node[gp node right] at (1.872,5.246) {$800000$};
\draw[gp path] (2.056,6.311)--(2.236,6.311);
\draw[gp path] (11.947,6.311)--(11.767,6.311);
\node[gp node right] at (1.872,6.311) {$1\times10^{6}$};
\draw[gp path] (2.056,7.376)--(2.236,7.376);
\draw[gp path] (11.947,7.376)--(11.767,7.376);
\node[gp node right] at (1.872,7.376) {$1.2\times10^{6}$};
\draw[gp path] (2.056,8.441)--(2.236,8.441);
\draw[gp path] (11.947,8.441)--(11.767,8.441);
\node[gp node right] at (1.872,8.441) {$1.4\times10^{6}$};
\draw[gp path] (2.056,0.985)--(2.056,1.165);
\draw[gp path] (2.056,8.441)--(2.056,8.261);
\node[gp node center] at (2.056,0.677) {$0$};
\draw[gp path] (2.955,0.985)--(2.955,1.165);
\draw[gp path] (2.955,8.441)--(2.955,8.261);
\node[gp node center] at (2.955,0.677) {$50$};
\draw[gp path] (3.854,0.985)--(3.854,1.165);
\draw[gp path] (3.854,8.441)--(3.854,8.261);
\node[gp node center] at (3.854,0.677) {$100$};
\draw[gp path] (4.754,0.985)--(4.754,1.165);
\draw[gp path] (4.754,8.441)--(4.754,8.261);
\node[gp node center] at (4.754,0.677) {$150$};
\draw[gp path] (5.653,0.985)--(5.653,1.165);
\draw[gp path] (5.653,8.441)--(5.653,8.261);
\node[gp node center] at (5.653,0.677) {$200$};
\draw[gp path] (6.552,0.985)--(6.552,1.165);
\draw[gp path] (6.552,8.441)--(6.552,8.261);
\node[gp node center] at (6.552,0.677) {$250$};
\draw[gp path] (7.451,0.985)--(7.451,1.165);
\draw[gp path] (7.451,8.441)--(7.451,8.261);
\node[gp node center] at (7.451,0.677) {$300$};
\draw[gp path] (8.350,0.985)--(8.350,1.165);
\draw[gp path] (8.350,8.441)--(8.350,8.261);
\node[gp node center] at (8.350,0.677) {$350$};
\draw[gp path] (9.249,0.985)--(9.249,1.165);
\draw[gp path] (9.249,8.441)--(9.249,8.261);
\node[gp node center] at (9.249,0.677) {$400$};
\draw[gp path] (10.149,0.985)--(10.149,1.165);
\draw[gp path] (10.149,8.441)--(10.149,8.261);
\node[gp node center] at (10.149,0.677) {$450$};
\draw[gp path] (11.048,0.985)--(11.048,1.165);
\draw[gp path] (11.048,8.441)--(11.048,8.261);
\node[gp node center] at (11.048,0.677) {$500$};
\draw[gp path] (11.947,0.985)--(11.947,1.165);
\draw[gp path] (11.947,8.441)--(11.947,8.261);
\node[gp node center] at (11.947,0.677) {$550$};
\draw[gp path] (2.056,8.441)--(2.056,0.985)--(11.947,0.985)--(11.947,8.441)--cycle;
\node[gp node center,rotate=-270] at (0.292,4.713) {$\frac{1}{v} / \frac{l\cdot s}{mol}$};
\node[gp node center] at (7.001,0.215) {$\frac{1}{c} / \frac{l}{mol}$};
\node[gp node right] at (7.944,8.107) {Obere Anfangsgeschwindigkeiten};
\gpcolor{rgb color={0.580,0.000,0.827}}
\gpsetpointsize{4.00}
\gp3point{gp mark 1}{}{(11.228,7.929)}
\gp3point{gp mark 1}{}{(5.113,3.238)}
\gp3point{gp mark 1}{}{(3.890,2.380)}
\gp3point{gp mark 1}{}{(2.416,1.943)}
\gp3point{gp mark 1}{}{(2.272,1.966)}
\gp3point{gp mark 1}{}{(8.586,8.107)}
\gpcolor{color=gp lt color border}
\node[gp node right] at (7.944,7.799) {Untere Anfangsgeschwindigkeiten};
\gpcolor{rgb color={0.000,0.620,0.451}}
\gp3point{gp mark 2}{}{(11.228,7.535)}
\gp3point{gp mark 2}{}{(5.113,3.068)}
\gp3point{gp mark 2}{}{(3.890,2.275)}
\gp3point{gp mark 2}{}{(2.416,1.876)}
\gp3point{gp mark 2}{}{(2.272,1.940)}
\gp3point{gp mark 2}{}{(8.586,7.799)}
\gpcolor{color=gp lt color border}
\node[gp node right] at (7.944,7.491) {Obere Ausgleichsgerade};
\gpcolor{rgb color={0.337,0.706,0.914}}
\draw[gp path] (8.128,7.491)--(9.044,7.491);
\draw[gp path] (2.272,1.633)--(2.362,1.695)--(2.453,1.757)--(2.543,1.819)--(2.634,1.880)%
  --(2.724,1.942)--(2.815,2.004)--(2.905,2.066)--(2.996,2.128)--(3.086,2.190)--(3.176,2.252)%
  --(3.267,2.314)--(3.357,2.376)--(3.448,2.438)--(3.538,2.499)--(3.629,2.561)--(3.719,2.623)%
  --(3.810,2.685)--(3.900,2.747)--(3.991,2.809)--(4.081,2.871)--(4.172,2.933)--(4.262,2.995)%
  --(4.352,3.056)--(4.443,3.118)--(4.533,3.180)--(4.624,3.242)--(4.714,3.304)--(4.805,3.366)%
  --(4.895,3.428)--(4.986,3.490)--(5.076,3.552)--(5.167,3.614)--(5.257,3.675)--(5.348,3.737)%
  --(5.438,3.799)--(5.528,3.861)--(5.619,3.923)--(5.709,3.985)--(5.800,4.047)--(5.890,4.109)%
  --(5.981,4.171)--(6.071,4.232)--(6.162,4.294)--(6.252,4.356)--(6.343,4.418)--(6.433,4.480)%
  --(6.524,4.542)--(6.614,4.604)--(6.704,4.666)--(6.795,4.728)--(6.885,4.790)--(6.976,4.851)%
  --(7.066,4.913)--(7.157,4.975)--(7.247,5.037)--(7.338,5.099)--(7.428,5.161)--(7.519,5.223)%
  --(7.609,5.285)--(7.700,5.347)--(7.790,5.408)--(7.881,5.470)--(7.971,5.532)--(8.061,5.594)%
  --(8.152,5.656)--(8.242,5.718)--(8.333,5.780)--(8.423,5.842)--(8.514,5.904)--(8.604,5.966)%
  --(8.695,6.027)--(8.785,6.089)--(8.876,6.151)--(8.966,6.213)--(9.057,6.275)--(9.147,6.337)%
  --(9.237,6.399)--(9.328,6.461)--(9.418,6.523)--(9.509,6.584)--(9.599,6.646)--(9.690,6.708)%
  --(9.780,6.770)--(9.871,6.832)--(9.961,6.894)--(10.052,6.956)--(10.142,7.018)--(10.233,7.080)%
  --(10.323,7.142)--(10.413,7.203)--(10.504,7.265)--(10.594,7.327)--(10.685,7.389)--(10.775,7.451)%
  --(10.866,7.513)--(10.956,7.575)--(11.047,7.637)--(11.137,7.699)--(11.228,7.761);
\gpcolor{color=gp lt color border}
\node[gp node right] at (7.944,7.183) {Untere Ausgleichgerade};
\gpcolor{rgb color={0.902,0.624,0.000}}
\draw[gp path] (8.128,7.183)--(9.044,7.183);
\draw[gp path] (2.272,1.587)--(2.362,1.646)--(2.453,1.704)--(2.543,1.762)--(2.634,1.821)%
  --(2.724,1.879)--(2.815,1.937)--(2.905,1.996)--(2.996,2.054)--(3.086,2.113)--(3.176,2.171)%
  --(3.267,2.229)--(3.357,2.288)--(3.448,2.346)--(3.538,2.404)--(3.629,2.463)--(3.719,2.521)%
  --(3.810,2.579)--(3.900,2.638)--(3.991,2.696)--(4.081,2.754)--(4.172,2.813)--(4.262,2.871)%
  --(4.352,2.929)--(4.443,2.988)--(4.533,3.046)--(4.624,3.104)--(4.714,3.163)--(4.805,3.221)%
  --(4.895,3.279)--(4.986,3.338)--(5.076,3.396)--(5.167,3.454)--(5.257,3.513)--(5.348,3.571)%
  --(5.438,3.629)--(5.528,3.688)--(5.619,3.746)--(5.709,3.804)--(5.800,3.863)--(5.890,3.921)%
  --(5.981,3.979)--(6.071,4.038)--(6.162,4.096)--(6.252,4.154)--(6.343,4.213)--(6.433,4.271)%
  --(6.524,4.329)--(6.614,4.388)--(6.704,4.446)--(6.795,4.505)--(6.885,4.563)--(6.976,4.621)%
  --(7.066,4.680)--(7.157,4.738)--(7.247,4.796)--(7.338,4.855)--(7.428,4.913)--(7.519,4.971)%
  --(7.609,5.030)--(7.700,5.088)--(7.790,5.146)--(7.881,5.205)--(7.971,5.263)--(8.061,5.321)%
  --(8.152,5.380)--(8.242,5.438)--(8.333,5.496)--(8.423,5.555)--(8.514,5.613)--(8.604,5.671)%
  --(8.695,5.730)--(8.785,5.788)--(8.876,5.846)--(8.966,5.905)--(9.057,5.963)--(9.147,6.021)%
  --(9.237,6.080)--(9.328,6.138)--(9.418,6.196)--(9.509,6.255)--(9.599,6.313)--(9.690,6.371)%
  --(9.780,6.430)--(9.871,6.488)--(9.961,6.546)--(10.052,6.605)--(10.142,6.663)--(10.233,6.721)%
  --(10.323,6.780)--(10.413,6.838)--(10.504,6.897)--(10.594,6.955)--(10.685,7.013)--(10.775,7.072)%
  --(10.866,7.130)--(10.956,7.188)--(11.047,7.247)--(11.137,7.305)--(11.228,7.363);
\gpcolor{color=gp lt color border}
\draw[gp path] (2.056,8.441)--(2.056,0.985)--(11.947,0.985)--(11.947,8.441)--cycle;
%% coordinates of the plot area
\gpdefrectangularnode{gp plot 1}{\pgfpoint{2.056cm}{0.985cm}}{\pgfpoint{11.947cm}{8.441cm}}
\end{tikzpicture}

\caption{Lineweaver-Burk Auftragung und Ausgleichgeraden}
\label{LB_Auftragung}
\end{figure}

Für das Erstellen der Abbildungen ist die Konzentration der angesetzten Lösungen zu berechnen.
Für Maßkolben eins ergibt sich:

 $$C_2 = \frac{\qty{0,1}{\mole\per\liter} \cdot \qty{0,001}{\liter}}{\qty{0,051}{\liter}} = \qty{0,00196}{\mole\per\liter}$$

Ermittelt wurden folgende Ausgleichsgeraden:

\begin{center}
\begin{tabular}{lcr}
$2310,38\cdot \displaystyle{\frac{1}{c}} + 93930 = \displaystyle{\frac{1}{v_0}}$ &
\hspace{5mm} &
$2177,73\cdot \displaystyle{\frac{1}{c}} + 86987 = \displaystyle{\frac{1}{v_0}}$\\
\end{tabular}
\end{center}

Für die Auswertung wurden die Werte des Maßkolbens mit \qty{2}{\milli\liter} Harstoff, nicht verwendet.
Diese liegen Klar ausserhalb der gesuchten Michaelis-Menten Beziehung und müssen daher Anderweitug Fehlerbehaftet sein.
Zurückzuführen ist dies liegt wahrscheinlich daran, dass zu viel Wasser in diesen Maßkolben gegeben wurde.
Wie der Durchführung zu entnehmen ist.
Weiter ergeben sich die in Tabelle \ref{erg1} dargestellten Ergebnisse.
\begin{table}[t]
\centering
\begin{tabular}{l|ccr}
	& Obergrenze & Untergrenze & MWA\\
\hline
\hline
$K_M$		& \qty{39.944}{\mole\per\liter} 		& \qty{40,656}{\mole\per\liter} 		& $\pm\qty{0,356}{\mole\per\liter}$\\
$V_{Max}$	& \qty{1,15e-5}{\mole\per\liter\per\second} 	& \qty{1,064e-5}{\mole\per\liter\per\second} 	& $\pm\qty{4,249e-7}{\mole\per\liter\per\second}$ \\
$k_2$		& \qty[per-mode=power]{17,185}{\per\second} 	& \qty[per-mode=power]{15,915}{\per\second} 	& $\pm$\qty[per-mode=power]{0,635}{\per\second}\\
\hline
\end{tabular}
\caption{Ergebnisse des ersten Versuchteils}
\label{erg1}
\end{table}


\subsection{Geschwindigkeitskonstante der katalysierten Harnstoffumsetzung}
Wie oben gezeigt, wurde die Urease Konzentration berechnet. 
Diese ist gleich für alle Versuchsteile.
In unserem Versuch wurde eine Urease Stammlösung von \qty{3,412e-5}{\mole\per\liter} angesetzt.
Nach Versuchsstart hatte die Lösung dann eine Urease Konzentration von \qty{6,69e-7}{\mole\per\liter}.
Verwendet wurden drei fehlerbehaftete Messgeräte. Die Waage hat einen Messfehler von \qty{1e-4}{\gram}, die Pipette hat einen Messfehler von \qty{3e-5}{\liter} und der Maßkolben einen Fehler von \qty{6e-5}{\liter}.

Es wurde zuerst ein Stammlösungsfehler von $C_U = \qty{1,037e-7}{\mole\per\liter}$ und ein Messlösungsfehler von $C_{U2} = \qty{2,281e-8}{\mole\per\liter}$ ermittelt.
Der Messlösungsfehler setzt sich aus dem Abwiegefehler und den jeweiligen Verdünnungsfehlern zusammen und wird wie folgt berechnet:
\begin{align*}
\Delta C_U &= \frac{1}{M} \cdot \sqrt{\left(\frac{1}{V}\cdot \Delta m \right)^2 + \left( - \frac{m}{V^2} \cdot \Delta V \right)^2}\\
\Delta C_U &= \frac{1}{\qty{6e5}{\gram \per \mole}} \cdot   \sqrt{
    \left(
        \frac{1}{\qty{0.01}{\liter}}\cdot \qty{1e-4}{\gram}
    \right)^2 +
   \left(
        -\frac{\qty{0.2047}{\gram}}{\qty{0.01}{\liter}^2} \cdot \qty{3e-5}{\liter}
    \right)^2
    }\\
\Delta C_U &= \qty{1,037e-7}{\mole\per\liter}\\
\Delta C_{U2} &= \sqrt{\left(\frac{1}{V_2}\cdot \sqrt{\left( V_1 \cdot \Delta C_1 \right)^2 + \left( C_1 \Delta V_1 \right)^2}\right)^2 +\left( - \frac{C_1 \cdot V_1}{V_2^2}\cdot \Delta V_2 \right)^2}\\
A&=\left(\frac{\sqrt{( \qty{0,01}{\liter} \cdot \qty{3,412e-5}{\mole\per\liter} )^2 + \left( \qty{3,412}{\mole\per\liter} \qty{1e-5}{\liter} \right)^2}}{\qty{0,051}{\liter}}\cdot \right)^2 \\
    B&=\left( - \frac{\qty{3,412e-5}{\mole\per\liter} \cdot\qty{0,01}{\liter}}{\qty{0,051}{\liter}^2}\cdot \qty{6e-5}{\liter} \right)^2\\
    \Delta C_{U2} &= \sqrt{ A+ B} = \qty{2,281e-8}{\mole\per\liter}\\
\end{align*}
Weiter kann nun über die oben errechnete Maximalgeschwindigkeit die Geschwindigkeitskonstante nach errechnet werden:
$$ k_2 = \frac{v_{Max}}{c_{U2}} = \frac{\qty{1,15}{\mole\per\liter\per\second}}{\qty{6,69}{\mole\per\liter}} = \qty[per-mode=power]{17,185}{\per\second}$$
Der Fehler wird wie folgt berechnet:
\begin{align*}
 \Delta k_2 &= \sqrt{\left( \frac{1}{C_{U2}}\cdot \Delta V_0 \right)^2 + \left( - \frac{V_0}{{C_{U2}}^2} \cdot \Delta C_{U2} \right)^2}\\
 \Delta k_2 &= \sqrt{\left( \frac{\qty{3.825e-7}{\mole \per \second \liter}}{\qty{6.69e-7}{\mole \per \liter}}  \right)^2 + \left( - \frac{\qty{1.126e-5}{\mole \per \second \liter}}{\qty{6.69e-7}{\mole \per \liter}}\cdot 2.281 \cdot 10^{-8} \frac{mol}{L} \right)^2}\\
 \Delta k_2 &= 0.572 s^{-1}
\end{align*}

Die Ergebnisse dieses Versuchsteils sind ebenfalls in Tabelle \ref{erg1} zu sehen.
Bewusst wurde hier die Mittelwertabweichung gewählt, anstelle des Einzelfehlers, da diese etwas größer als der Einzelfehler ist, sich aber noch im angemessenen Rahmen bewegt.


\subsection{Geschwindigkeitskonstante der unkatalysierten Harnstoffumsetzung}
Verwendet wurde hier eine \qty{0,5}{\mole\per\liter} Harnstofflösung.
Deren Fehler wird als 0 angenommen.
Wie in den vorherigen Versuchen wird auch hier das Leitfähigkeitsmessgerät ebenfalls als fehlerfrei angesehen.
Unsere errechnete Geschwindigkeitskonstante ist in diesem Fall ohne Fehler.

Für die Berechnung der Geschwindigkeitskonstante wird erst wie oben beschrieben aus der gemessenen Leitfähigkeit die Konzentration errechnet (Gleichung \ref{UmrechnungLeitfähigkeitKonzentration}).
Alle Konzentrationen werden anschließend durch die erste gemessene Konzentration geteilt. 
Danach wird der natürliche Logarithmus davon ermittelt.
Es sind Werte in dem Bereich 0 bis 0,4 ermittelt worden, welche anschließend in Abhängigkeit der Zeit dargestellt worden sind.
Eine Ausgleichsgerade durch die errechneten Werte gibt nach dem Geschwindigkeitsgesetz erster Ordnung (Gleichung \ref{gesch.gesetz}) mit seiner Steigung die Geschwindigkeitskonstante der Reaktion an.
\begin{figure}[t]
\centering
\begin{tikzpicture}[gnuplot]
%% generated with GNUPLOT 5.4p5 (Lua 5.4; terminal rev. Jun 2020, script rev. 115)
%% Sa 19 Nov 2022 18:38:27 CET
\path (0.000,0.000) rectangle (12.500,8.750);
\gpcolor{color=gp lt color border}
\gpsetlinetype{gp lt border}
\gpsetdashtype{gp dt solid}
\gpsetlinewidth{1.00}
\draw[gp path] (1.504,0.985)--(1.684,0.985);
\draw[gp path] (11.947,0.985)--(11.767,0.985);
\node[gp node right] at (1.320,0.985) {$0$};
\draw[gp path] (1.504,1.917)--(1.684,1.917);
\draw[gp path] (11.947,1.917)--(11.767,1.917);
\node[gp node right] at (1.320,1.917) {$0.05$};
\draw[gp path] (1.504,2.849)--(1.684,2.849);
\draw[gp path] (11.947,2.849)--(11.767,2.849);
\node[gp node right] at (1.320,2.849) {$0.1$};
\draw[gp path] (1.504,3.781)--(1.684,3.781);
\draw[gp path] (11.947,3.781)--(11.767,3.781);
\node[gp node right] at (1.320,3.781) {$0.15$};
\draw[gp path] (1.504,4.713)--(1.684,4.713);
\draw[gp path] (11.947,4.713)--(11.767,4.713);
\node[gp node right] at (1.320,4.713) {$0.2$};
\draw[gp path] (1.504,5.645)--(1.684,5.645);
\draw[gp path] (11.947,5.645)--(11.767,5.645);
\node[gp node right] at (1.320,5.645) {$0.25$};
\draw[gp path] (1.504,6.577)--(1.684,6.577);
\draw[gp path] (11.947,6.577)--(11.767,6.577);
\node[gp node right] at (1.320,6.577) {$0.3$};
\draw[gp path] (1.504,7.509)--(1.684,7.509);
\draw[gp path] (11.947,7.509)--(11.767,7.509);
\node[gp node right] at (1.320,7.509) {$0.35$};
\draw[gp path] (1.504,8.441)--(1.684,8.441);
\draw[gp path] (11.947,8.441)--(11.767,8.441);
\node[gp node right] at (1.320,8.441) {$0.4$};
\draw[gp path] (1.504,0.985)--(1.504,1.165);
\draw[gp path] (1.504,8.441)--(1.504,8.261);
\node[gp node center] at (1.504,0.677) {$0$};
\draw[gp path] (3.245,0.985)--(3.245,1.165);
\draw[gp path] (3.245,8.441)--(3.245,8.261);
\node[gp node center] at (3.245,0.677) {$100$};
\draw[gp path] (4.985,0.985)--(4.985,1.165);
\draw[gp path] (4.985,8.441)--(4.985,8.261);
\node[gp node center] at (4.985,0.677) {$200$};
\draw[gp path] (6.726,0.985)--(6.726,1.165);
\draw[gp path] (6.726,8.441)--(6.726,8.261);
\node[gp node center] at (6.726,0.677) {$300$};
\draw[gp path] (8.466,0.985)--(8.466,1.165);
\draw[gp path] (8.466,8.441)--(8.466,8.261);
\node[gp node center] at (8.466,0.677) {$400$};
\draw[gp path] (10.207,0.985)--(10.207,1.165);
\draw[gp path] (10.207,8.441)--(10.207,8.261);
\node[gp node center] at (10.207,0.677) {$500$};
\draw[gp path] (11.947,0.985)--(11.947,1.165);
\draw[gp path] (11.947,8.441)--(11.947,8.261);
\node[gp node center] at (11.947,0.677) {$600$};
\draw[gp path] (1.504,8.441)--(1.504,0.985)--(11.947,0.985)--(11.947,8.441)--cycle;
\node[gp node center,rotate=-270] at (0.292,4.713) {$\ln{\frac{c_t}{c_0}}$};
\node[gp node center] at (6.725,0.215) {t / s};
\node[gp node right] at (4.632,8.107) {Data};
\gpcolor{rgb color={0.580,0.000,0.827}}
\gpsetpointsize{4.00}
\gp3point{gp mark 1}{}{(1.591,0.985)}
\gp3point{gp mark 1}{}{(1.678,0.985)}
\gp3point{gp mark 1}{}{(1.765,0.985)}
\gp3point{gp mark 1}{}{(1.852,0.985)}
\gp3point{gp mark 1}{}{(1.939,1.309)}
\gp3point{gp mark 1}{}{(2.026,1.309)}
\gp3point{gp mark 1}{}{(2.113,1.309)}
\gp3point{gp mark 1}{}{(2.200,1.309)}
\gp3point{gp mark 1}{}{(2.287,1.628)}
\gp3point{gp mark 1}{}{(2.374,1.628)}
\gp3point{gp mark 1}{}{(2.461,1.628)}
\gp3point{gp mark 1}{}{(2.548,1.628)}
\gp3point{gp mark 1}{}{(2.635,1.628)}
\gp3point{gp mark 1}{}{(2.722,1.941)}
\gp3point{gp mark 1}{}{(2.809,1.941)}
\gp3point{gp mark 1}{}{(2.896,1.941)}
\gp3point{gp mark 1}{}{(2.983,2.249)}
\gp3point{gp mark 1}{}{(3.070,2.249)}
\gp3point{gp mark 1}{}{(3.157,2.249)}
\gp3point{gp mark 1}{}{(3.245,2.249)}
\gp3point{gp mark 1}{}{(3.332,2.249)}
\gp3point{gp mark 1}{}{(3.419,2.552)}
\gp3point{gp mark 1}{}{(3.506,2.552)}
\gp3point{gp mark 1}{}{(3.593,2.552)}
\gp3point{gp mark 1}{}{(3.680,2.552)}
\gp3point{gp mark 1}{}{(3.767,2.552)}
\gp3point{gp mark 1}{}{(3.854,2.851)}
\gp3point{gp mark 1}{}{(3.941,2.851)}
\gp3point{gp mark 1}{}{(4.028,2.851)}
\gp3point{gp mark 1}{}{(4.115,2.851)}
\gp3point{gp mark 1}{}{(4.202,3.144)}
\gp3point{gp mark 1}{}{(4.289,3.144)}
\gp3point{gp mark 1}{}{(4.376,3.144)}
\gp3point{gp mark 1}{}{(4.463,3.144)}
\gp3point{gp mark 1}{}{(4.550,3.433)}
\gp3point{gp mark 1}{}{(4.637,3.433)}
\gp3point{gp mark 1}{}{(4.724,3.433)}
\gp3point{gp mark 1}{}{(4.811,3.433)}
\gp3point{gp mark 1}{}{(4.898,3.433)}
\gp3point{gp mark 1}{}{(4.985,3.718)}
\gp3point{gp mark 1}{}{(5.072,3.718)}
\gp3point{gp mark 1}{}{(5.159,3.718)}
\gp3point{gp mark 1}{}{(5.246,3.718)}
\gp3point{gp mark 1}{}{(5.333,3.998)}
\gp3point{gp mark 1}{}{(5.420,3.998)}
\gp3point{gp mark 1}{}{(5.507,3.998)}
\gp3point{gp mark 1}{}{(5.594,3.998)}
\gp3point{gp mark 1}{}{(5.681,4.274)}
\gp3point{gp mark 1}{}{(5.768,4.274)}
\gp3point{gp mark 1}{}{(5.855,4.274)}
\gp3point{gp mark 1}{}{(5.942,4.274)}
\gp3point{gp mark 1}{}{(6.029,4.274)}
\gp3point{gp mark 1}{}{(6.116,4.274)}
\gp3point{gp mark 1}{}{(6.203,4.546)}
\gp3point{gp mark 1}{}{(6.290,4.546)}
\gp3point{gp mark 1}{}{(6.377,4.546)}
\gp3point{gp mark 1}{}{(6.464,4.814)}
\gp3point{gp mark 1}{}{(6.551,4.814)}
\gp3point{gp mark 1}{}{(6.638,4.814)}
\gp3point{gp mark 1}{}{(6.726,4.814)}
\gp3point{gp mark 1}{}{(6.813,4.814)}
\gp3point{gp mark 1}{}{(6.900,5.079)}
\gp3point{gp mark 1}{}{(6.987,5.079)}
\gp3point{gp mark 1}{}{(7.074,5.079)}
\gp3point{gp mark 1}{}{(7.161,5.079)}
\gp3point{gp mark 1}{}{(7.248,5.079)}
\gp3point{gp mark 1}{}{(7.335,5.340)}
\gp3point{gp mark 1}{}{(7.422,5.340)}
\gp3point{gp mark 1}{}{(7.509,5.340)}
\gp3point{gp mark 1}{}{(7.596,5.340)}
\gp3point{gp mark 1}{}{(7.683,5.597)}
\gp3point{gp mark 1}{}{(7.770,5.597)}
\gp3point{gp mark 1}{}{(7.857,5.597)}
\gp3point{gp mark 1}{}{(7.944,5.597)}
\gp3point{gp mark 1}{}{(8.031,5.850)}
\gp3point{gp mark 1}{}{(8.118,5.850)}
\gp3point{gp mark 1}{}{(8.205,5.850)}
\gp3point{gp mark 1}{}{(8.292,5.850)}
\gp3point{gp mark 1}{}{(8.379,5.850)}
\gp3point{gp mark 1}{}{(8.466,6.101)}
\gp3point{gp mark 1}{}{(8.553,6.101)}
\gp3point{gp mark 1}{}{(8.640,6.101)}
\gp3point{gp mark 1}{}{(8.727,6.101)}
\gp3point{gp mark 1}{}{(8.814,6.347)}
\gp3point{gp mark 1}{}{(8.901,6.347)}
\gp3point{gp mark 1}{}{(8.988,6.347)}
\gp3point{gp mark 1}{}{(9.075,6.347)}
\gp3point{gp mark 1}{}{(9.162,6.591)}
\gp3point{gp mark 1}{}{(9.249,6.591)}
\gp3point{gp mark 1}{}{(9.336,6.591)}
\gp3point{gp mark 1}{}{(9.423,6.591)}
\gp3point{gp mark 1}{}{(9.510,6.591)}
\gp3point{gp mark 1}{}{(9.597,6.591)}
\gp3point{gp mark 1}{}{(9.684,6.832)}
\gp3point{gp mark 1}{}{(9.771,6.832)}
\gp3point{gp mark 1}{}{(9.858,6.832)}
\gp3point{gp mark 1}{}{(9.945,7.069)}
\gp3point{gp mark 1}{}{(10.032,7.069)}
\gp3point{gp mark 1}{}{(10.119,7.069)}
\gp3point{gp mark 1}{}{(10.207,7.069)}
\gp3point{gp mark 1}{}{(10.294,7.069)}
\gp3point{gp mark 1}{}{(10.381,7.304)}
\gp3point{gp mark 1}{}{(10.468,7.304)}
\gp3point{gp mark 1}{}{(10.555,7.304)}
\gp3point{gp mark 1}{}{(10.642,7.304)}
\gp3point{gp mark 1}{}{(10.729,7.304)}
\gp3point{gp mark 1}{}{(10.816,7.535)}
\gp3point{gp mark 1}{}{(10.903,7.535)}
\gp3point{gp mark 1}{}{(10.990,7.535)}
\gp3point{gp mark 1}{}{(11.077,7.535)}
\gp3point{gp mark 1}{}{(11.164,7.764)}
\gp3point{gp mark 1}{}{(11.251,7.764)}
\gp3point{gp mark 1}{}{(11.338,7.764)}
\gp3point{gp mark 1}{}{(11.425,7.764)}
\gp3point{gp mark 1}{}{(11.512,7.990)}
\gp3point{gp mark 1}{}{(11.599,7.990)}
\gp3point{gp mark 1}{}{(11.686,7.990)}
\gp3point{gp mark 1}{}{(11.773,7.990)}
\gp3point{gp mark 1}{}{(11.860,7.990)}
\gp3point{gp mark 1}{}{(11.947,8.213)}
\gp3point{gp mark 1}{}{(5.274,8.107)}
\gpcolor{color=gp lt color border}
\node[gp node right] at (4.632,7.799) {Ausgleichsgerade};
\gpcolor{rgb color={0.000,0.620,0.451}}
\draw[gp path] (4.816,7.799)--(5.732,7.799);
\draw[gp path] (1.504,1.117)--(1.609,1.190)--(1.715,1.263)--(1.820,1.336)--(1.926,1.409)%
  --(2.031,1.482)--(2.137,1.555)--(2.242,1.628)--(2.348,1.701)--(2.453,1.774)--(2.559,1.847)%
  --(2.664,1.921)--(2.770,1.994)--(2.875,2.067)--(2.981,2.140)--(3.086,2.213)--(3.192,2.286)%
  --(3.297,2.359)--(3.403,2.432)--(3.508,2.505)--(3.614,2.578)--(3.719,2.651)--(3.825,2.725)%
  --(3.930,2.798)--(4.036,2.871)--(4.141,2.944)--(4.247,3.017)--(4.352,3.090)--(4.458,3.163)%
  --(4.563,3.236)--(4.669,3.309)--(4.774,3.382)--(4.880,3.456)--(4.985,3.529)--(5.090,3.602)%
  --(5.196,3.675)--(5.301,3.748)--(5.407,3.821)--(5.512,3.894)--(5.618,3.967)--(5.723,4.040)%
  --(5.829,4.113)--(5.934,4.186)--(6.040,4.260)--(6.145,4.333)--(6.251,4.406)--(6.356,4.479)%
  --(6.462,4.552)--(6.567,4.625)--(6.673,4.698)--(6.778,4.771)--(6.884,4.844)--(6.989,4.917)%
  --(7.095,4.990)--(7.200,5.064)--(7.306,5.137)--(7.411,5.210)--(7.517,5.283)--(7.622,5.356)%
  --(7.728,5.429)--(7.833,5.502)--(7.939,5.575)--(8.044,5.648)--(8.150,5.721)--(8.255,5.794)%
  --(8.361,5.868)--(8.466,5.941)--(8.571,6.014)--(8.677,6.087)--(8.782,6.160)--(8.888,6.233)%
  --(8.993,6.306)--(9.099,6.379)--(9.204,6.452)--(9.310,6.525)--(9.415,6.598)--(9.521,6.672)%
  --(9.626,6.745)--(9.732,6.818)--(9.837,6.891)--(9.943,6.964)--(10.048,7.037)--(10.154,7.110)%
  --(10.259,7.183)--(10.365,7.256)--(10.470,7.329)--(10.576,7.402)--(10.681,7.476)--(10.787,7.549)%
  --(10.892,7.622)--(10.998,7.695)--(11.103,7.768)--(11.209,7.841)--(11.314,7.914)--(11.420,7.987)%
  --(11.525,8.060)--(11.631,8.133)--(11.736,8.206)--(11.842,8.280)--(11.947,8.353);
\gpcolor{color=gp lt color border}
\draw[gp path] (1.504,8.441)--(1.504,0.985)--(11.947,0.985)--(11.947,8.441)--cycle;
%% coordinates of the plot area
\gpdefrectangularnode{gp plot 1}{\pgfpoint{1.504cm}{0.985cm}}{\pgfpoint{11.947cm}{8.441cm}}
\end{tikzpicture}
%% gnuplot variables

\caption{Logarithmische Auftragung der relativen Konzentrationsänderung über die Zeit}
\label{lnWerte}
\end{figure}
In Abbildung \ref{lnWerte} sind unsere errechneten Werte sowie die Ausgleichsgerade eingezeichnet.
Ermittelt wurde eine Geschwindigkeitskonstante von $0,000647\ s^{-1}$ 

\printbibliography
\end{document}
